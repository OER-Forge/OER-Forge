\section{About OER-Forge}\label{about-oer-forge}

OER-Forge is an open-source project dedicated to building accessible,
high-quality, and modern educational resources for physics and
computational science. Our mission is to empower educators, students,
and lifelong learners by providing free, adaptable, and open
materials---while supporting a transparent, collaborative, and
sustainable approach to educational publishing.

Below is a plot of the rising cost of textbooks over time. It shows a
2.8 times the rate of inflation increase in textbooks costs. And it
shows a nearly 1400\% increase in real dollars over the time time.

Full disclosure. This plot is from
\href{https://myelearningworld.com/textbook-prices-vs-inflation/}{My
eLearning World}, which advocates for eLearning solutions. But the costs
of those are still borne by students. Both textbooks and eLearning
solutions are often behind paywalls, expensive, and not accessible to
all students.

This is where
\href{https://en.wikipedia.org/wiki/Open_educational_resources}{Open
Educational Resources (OERs)} come in. OERs are free, openly licensed
educational materials that can be used, adapted, and shared by anyone.
They provide an alternative to traditional textbooks and eLearning
solutions, making education more affordable and accessible.

\subsection{Our Mission}\label{our-mission}

\begin{itemize}
\tightlist
\item
  \textbf{Accessibility:} We design all resources to be usable by
  everyone, including people with disabilities, and available in
  multiple formats (HTML, PDF, Markdown, LaTeX, Jupyter, and more).
\item
  \textbf{Openness:} Everything is
  \href{https://github.com/OER-Forge}{open source} and
  \href{https://github.com/OER-Forge/OER-Forge/blob/main/LICENSE}{free
  to use, modify, and share for non-commercial use}. We challenge the
  corporate control of educational content and promote equitable access
  for all.
\item
  \textbf{Collaboration:} OER-Forge is a community effort---anyone can
  contribute, whether you're an educator, student, developer, or
  enthusiast. \href{https://github.com/OER-Forge/OER-Forge/pulls}{Pull
  requests},
  \href{https://github.com/OER-Forge/OER-Forge/issues}{suggestions}, and
  \href{mailto:hello@oerforge.org}{feedback} are always welcome.
\item
  \textbf{Simplicity:} Our sites and materials are intentionally simple,
  semantic, and easy to use, modify, and distribute, ensuring broad
  compatibility and accessibility.
\item
  \textbf{Transparency:} The build process, content, and code are open
  and documented for contributors and users alike.
\end{itemize}

\subsection{Who We Are}\label{who-we-are}

OER-Forge is being developed by \href{https://dannycab.github.io/}{Danny
Caballero}, professor of physics and computational science at Michigan
State University. The project is inspired by years of teaching,
learning, and collaborating with others who share a passion for open,
equitable education.

\subsection{Projects}\label{projects}

\begin{itemize}
\tightlist
\item
  \textbf{Modern Classical Mechanics:} The first major course site built
  with OER-Forge, providing interactive, accessible resources for
  teaching and learning classical mechanics.
  \href{https://dannycaballero.info/modern-classical-mechanics/}{Visit
  Modern Classical Mechanics}
\item
  \textbf{Future Courses:} OER-Forge is designed to support many classes
  and topics. Modern Classical Mechanics is just the beginning---more
  courses and resources are coming soon.
\end{itemize}

\subsection{Philosophy \& Design
Principles}\label{philosophy-design-principles}

\begin{enumerate}
\def\labelenumi{\arabic{enumi}.}
\tightlist
\item
  \textbf{Open source and free:} All code and content are open and
  freely available.
\item
  \textbf{Accessibility first:} We follow best practices and continually
  improve to meet WCAG and other standards.
\item
  \textbf{Community-driven:} Anyone can contribute---suggest changes,
  add content, or improve materials.
\item
  \textbf{Simple, semantic, and accessible design:} We use semantic
  HTML, ARIA roles, and keyboard navigation to ensure usability for all.
\item
  \textbf{Transparent and documented:} The build process and project
  management are open for all contributors.
\end{enumerate}

\subsection{Accessibility Commitments}\label{accessibility-commitments}

\begin{itemize}
\tightlist
\item
  Ongoing work to ensure all materials are accessible and usable by
  people with disabilities
\item
  Keyboard navigation and screen reader compatibility
\item
  High color contrast and readable font sizes
\item
  Descriptive alt text and link text
\item
  Open build process and documentation for contributors
\end{itemize}

\subsection{How to Contribute}\label{how-to-contribute}

We welcome contributions from anyone interested in improving open
educational resources: - Suggest new content or improvements - Report
issues or bugs - Review or submit pull requests - Add new activities,
simulations, or resources - Help improve accessibility and design

You can contribute by creating an issue or pull request on our
\href{https://github.com/open-physics-ed/open-physics-ed-org.github.io}{GitHub
repository}.

\subsection{Contact \& Community}\label{contact-community}

\begin{itemize}
\tightlist
\item
  \href{https://github.com/OER-Forge}{OER Forge on GitHub}
\item
  \href{https://bsky.app/profile/oerforge.org}{OER Forge on Bluesky}
\item
  \href{mailto:hello@oerforge.org}{Email}
\item
  \href{https://www.buymeacoffee.com/dannycab}{Buy us a coffee}
\end{itemize}

\begin{center}\rule{0.5\linewidth}{0.5pt}\end{center}

OER Forge is open source and always evolving. Join us in building a more
accessible, equitable future for physics education and open science.
