\section{Introduction to Newton's Laws of
Motion}\label{introduction-to-newtons-laws-of-motion}

Physics is the study of matter, energy, and the interactions between
them. In this lesson, we'll explore \textbf{Newton's Laws of Motion}.

\begin{center}\rule{0.5\linewidth}{0.5pt}\end{center}

\subsection{Table of Contents}\label{table-of-contents}

\begin{enumerate}
\def\labelenumi{\arabic{enumi}.}
\tightlist
\item
  \hyperref[newtons-first-law]{Newton's First Law}
\item
  \hyperref[newtons-second-law]{Newton's Second Law}
\item
  \hyperref[newtons-third-law]{Newton's Third Law}
\item
  \hyperref[sample-problems]{Sample Problems}
\item
  \hyperref[references]{References}
\end{enumerate}

\begin{center}\rule{0.5\linewidth}{0.5pt}\end{center}

\subsection{Newton's First Law}\label{newtons-first-law}

\begin{quote}
An object at rest stays at rest and an object in motion stays in motion
unless acted upon by an external force.
\end{quote}

\begin{itemize}
\tightlist
\item
  Also known as the \textbf{Law of Inertia}.
\item
  Example: A book on a table remains at rest unless pushed.
\end{itemize}

\begin{center}\rule{0.5\linewidth}{0.5pt}\end{center}

\subsection{Newton's Second Law}\label{newtons-second-law}

Newton's Second Law relates force, mass, and acceleration:

\[
F = m \cdot a
\]

Where: - \(F\) = Force (Newtons) - \(m\) = Mass (kg) - \(a\) =
Acceleration (\(m/s^2\))

\textbf{Example Calculation:}
